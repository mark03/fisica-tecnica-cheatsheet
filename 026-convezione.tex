\section{Convezione}
Trasferimento di energia tra una superficie solida e un fluido adiacente in movimento.
\begin{description}
 \item[Convezione forzata] Avviene quando il fluido è forzato a scorrere su una superficie da mezzi esterni
 \item[Convezione naturale] Avviene quando il moto del fluido è causato da forze ascensionali che sono indotte dalle differenze di densità dovute alla variazione di temperatura del fluido in un campo gravitazionale
\end{description}

\subsection{Equazione generale}
\[ \dot{q} = h \cdot \qty(T_s - T_f) \]
\begin{align*}
h & \quad \text{coefficiente convettivo} \\
T_s & \quad \text{temperatura superficie} \\
T_f & \quad \text{temperatura fluido} \\
\end{align*}
\[ \qq*{Ponendo} R_{conv} = \frac{1}{h \cdot A} \qq{si ha che} \dot{Q}_{conv} = -\frac{\Delta T}{R_{conv}} \]

\subsection{Raggio critico di isolamento}
Considerando gusci cilindrici aventi come utimo strato uno strato di isolante soggetto a fenomeni convettivi.
\[ R_{tot} = R_{isol} + R_{conv} = \frac{\ln(\frac{R_e}{R_i})}{2\pi L k} + \frac{1}{2\pi R_e L h} \]
\begin{align*}
R_e & \quad \text{raggio esterno isolante} \\
R_i & \quad \text{raggio interno isolante}
\end{align*}
Si ha che inizialmente $R_{tot}$ decresce all'aumentare dello spessore di isolante.
Si ha $R_{tot}$ minimo quando $R_e = R_{critico} = \frac{k}{h}$.

\subsection{Coefficiente convettivo}

\begin{tabular}{p{5cm}r}
    \toprule
    Fluido & $h \quad \qty[\frac{W}{m^2K}]$ \\ \midrule
    gas stagnante &  \numrange{5}{50} \\
    acqua stagnante & $100$ \\
    gas in moto & \numrange{50}{1000} \\
    olio minerale & \numrange{50}{3000} \\
    acqua in mto & \numrange{200}{10000} \\
    acqua ebollizione/condensazione & \numrange{1000}{100000} \\
    metalli liquidi & \numrange{10000}{100000} \\
    \bottomrule
\end{tabular}

Il coefficiente convettivo non è una proprietà della materia e dipende da:
\begin{align*}
    \rho & \quad \text{densità} \\
    \mu & \quad \text{viscosità} \\
    c_p & \quad \text{cal. spec. pressione cost.} \\
    k & \quad \text{conduttività} \\
    w & \quad \text{velocità} \\
    D & \quad \text{diametro equivalente} \\
\end{align*}


\subsection{Convezione forzata}
\subsubsection{Numero di Nusselt}
\[ \Nus = \frac{h D}{k} \qty( = \frac{h \Delta T}{\frac{k}{D} \Delta T})\] 
Può essere interpretato come rapporto tra potenza termica scambiata per convezione e potenza termica scambiata per conduzione nelo strato limite.

\subsubsection{Numero di Reynolds}
\[ \Rey = \frac{\rho w D}{\mu} \qty( = \frac{F_{inerzia}}{F_{viscose}})\] 
Può essere interpretato come rapporto tra la risultante delle forze di inerzia e la risultante delle forze viscose.

\textbf{Valori tipici di Reynolds}\\
\textit{Moto in un condotto:}\\
\phantom{→}$\put(3,2){\line(0,1){5}}\to$ laminare se $\Rey < 2000$, turbolento se $\Rey > 2500$\\
\textit{Moto lungo lastra piana:}\\
\phantom{→}$\put(3,2){\line(0,1){5}}\to$ laminare se $\Rey < 5 \cdot 10^5$, turbolento se $\Rey > 5 \cdot 10^5$\\
\textit{Moto attorno ad un cilindro:}\\
\phantom{→}$\put(3,2){\line(0,1){5}}\to$ laminare se $\Rey < 2 \cdot 10^5$, turbolento se $\Rey > 2 \cdot 10^5$\\

\subsubsection{Numero di Prandtl}
\[ \Pra = \frac{c_p \mu}{k} \]
Si ottiene dividendo $\nu = \frac{\mu}{\rho}$ (viscosità cinematica, $\nu$) per $\frac{k}{\rho c_p}$ (diffusività termica).

\textbf{Valori tipici di Prandtl}\\
\begin{tabular}{p{3cm}r}
    \toprule
    gas ideali & $0.7$\\
    acqua & \numrange{2}{10}\\
    metalli liquidi & \numrange{0.005}{0.03}\\
    \bottomrule
\end{tabular}
