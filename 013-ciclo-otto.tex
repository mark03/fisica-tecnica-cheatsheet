\section{Ciclo Otto}
Ciclo simmetrico cosituito da due \emph{isoentropiche} e due \emph{isocore}.
Applicazioni: prevalentemente in campo automobilistico. $r_v$ tra 6 e 10 per evitare autocombustione della miscela in $1-2$.

\begin{minipage}{.5\linewidth}
    \begin{tikzpicture}[thick,>=stealth']
        \coordinate (O) at (0,0);
        \draw[->] (0,0) -- (3.5,0) coordinate[label = {below:$s$}] (xmax);
        \draw[->] (0,0) -- (0,3.5) coordinate[label = {left:$T$}] (ymax);
        
        \draw (1,1) node[below left] {1} -- (1,1.3) node[above left] {2} parabola (3,3) node[above right] {3} -- (3,2) node[below] {4};
        \draw (1,1) parabola (3,2);
    \end{tikzpicture}
\end{minipage}%
\begin{minipage}{.5\linewidth}
    \begin{tikzpicture}[thick,>=stealth']
        \coordinate (O) at (0,0);
        \draw[->] (0,0) -- (3.5,0) coordinate[label = {below:$v$}] (xmax);
        \draw[->] (0,0) -- (0,3.5) coordinate[label = {left:$P$}] (ymax);
    
        \draw (3.2,0.5) node[right] {1} parabola (0.7,1) node[left] {2} -- (0.7,3) node[left] {3} parabola bend (3.2,1) (3.2,1) node[right] {4} -- cycle;
    \end{tikzpicture}
\end{minipage}

Il rendimento del ciclo Otto nel caso di \emph{gas perfetto}:
\[
    \eta_{\text{Otto}} = 1 - \frac{q_f}{q_c} = 1 - \frac{T_4-T_1}{T_3-T_2} = 1 - \frac{T_1}{T_2}
\]

Dal bilancio entropico tra 1 e 2 vale:
\[
    \left( \frac{T_2}{T_1} \right)^{c_v} = \left( \frac{V_1}{V_2} \right)^{R^*} 
\]
Chiamando rapporto di compressione volumentrico $r_v = \frac{V_1}{V_2}$
\[
    \eta_{\text{Otto}} = 1 - r_v^{1-k}
\]

\subsection{Lavoro specifico per il ciclo Otto}
\[
    l = c_v(T_3-T_4) + c_v(T_2-T_1) = c_vT_3\left(1-\frac{1}{r_v^{k-1}}\right) - c_vT_1(r_v^{k-1}-1)
\]
\[
    r_v^{\text{opt}} = \left(\frac{T_3}{T_1}\right)^{\frac{1}{2(k-1)}}
\]
