\documentclass[10pt,landscape]{article}
\usepackage[italian]{babel}
\usepackage[utf8]{inputenc}
\usepackage{multicol}
\usepackage{multirow}
\usepackage{calc}
\usepackage{ifthen}
\usepackage[landscape]{geometry}
\usepackage{hyperref}
\usepackage{amsmath}
\usepackage{amssymb}
\usepackage{tabularx}
\usepackage{caption}
\usepackage{verbatim}
\usepackage{systeme}
\usepackage{nicefrac}
\usepackage{accents}
\usepackage{enumitem}
\usepackage[printwatermark]{xwatermark}
\usepackage{tikz}
\usetikzlibrary{calc,matrix,arrows,intersections,decorations.pathreplacing,shapes.misc}
\usepackage[compact]{titlesec}
\usepackage{microtype}
\usepackage[flushleft]{threeparttable}
\usepackage{textcomp}
\usepackage{pifont}
\usepackage{pgfplots}
\usepackage[arrowdel]{physics}
\usepackage{siunitx}
\usepackage{booktabs}
\usepackage{stmaryrd}
\usepackage{chemformula}

% Ranges of numbers
\sisetup{range-phrase=--}

% This sets page margins to .5 inch if using letter paper, and to 1cm
% if using A4 paper. (This probably isn't strictly necessary.)
% If using another size paper, use default 1cm margins.
\ifthenelse{\lengthtest { \paperwidth = 11in}}
{ \geometry{top=.5in,left=.5in,right=.5in,bottom=.5in} }
{\ifthenelse{ \lengthtest{ \paperwidth = 297mm}}
	{\geometry{top=1cm,left=1cm,right=1cm,bottom=1cm} }
	{\geometry{top=1cm,left=1cm,right=1cm,bottom=1cm} }
}

% Turn off header and footer
\pagestyle{empty}

% Reduce size of \section e \subsection
\titleformat{\section}{\normalfont\large\bfseries}{\thesection}{1em}{}
\titleformat{\subsection}{\normalfont\normalsize\bfseries}{\thesubsection}{1em}{}
\titlespacing{\section}{0pt}{0ex}{-0.5ex}
\titlespacing{\subsection}{0pt}{0ex}{-0.5ex}

% Define BibTeX command
\def\BibTeX{{\rm B\kern-.05em{\sc i\kern-.025em b}\kern-.08em
		T\kern-.1667em\lower.7ex\hbox{E}\kern-.125emX}}

% Don't print section numbers
\setcounter{secnumdepth}{0}

\setlength{\parindent}{0pt}
\setlength{\parskip}{0pt plus 0.5ex}

\setlist[itemize]{noitemsep, nolistsep}

% No idea, without this tikz gives a warning
\pgfplotsset{compat=1.16}

\pgfmathdeclarefunction{gauss}{2}{%
  \pgfmathparse{1/(#2*sqrt(2*pi))*exp(-((x-#1)^2)/(2*#2^2))}%
}

\newcommand{\sla}{\shortleftarrow}
\newcommand{\sra}{\shortrightarrow}

\DeclareMathOperator{\Nus}{Nu}
\DeclareMathOperator{\Rey}{Re}
\DeclareMathOperator{\Pra}{Pr}
\DeclareMathOperator{\Gra}{Gr}
\DeclareMathOperator{\Pec}{Pe}
\DeclareMathOperator{\Ray}{Ra}

\newcommand{\gaussiana}[4]{
	\begin{axis}[
		hide axis,
		axis lines=left,
		xtick=\empty,
		ytick=\empty,
		every axis plot post/.append style={mark=none,domain=-1:1,samples=50,smooth},
		xmin=#1,
		xmax=#2,
		ymin=#3,
		ymax=#4,
		width=5cm
	]
		\addplot[black] {gauss(0,0.6)};
	\end{axis}
}

\usepackage{lipsum}

\begin{document}

\raggedright
\footnotesize
\begin{multicols}{3}

% multicol parameters
% These lengths are set only within the two main columns
%\setlength{\columnseprule}{0.25pt}
\setlength{\premulticols}{1pt}
\setlength{\postmulticols}{1pt}
\setlength{\multicolsep}{1pt}
\setlength{\columnsep}{2pt}

{\Large{\textbf{Fisica Tecnica}}}

\section{Concetti base}
\subsection{Sistema termodinamico}
\begin{tabular}{lccc}
    \toprule
    & Calore & Lavoro & Massa\\ \midrule
    Adiabatico & no & & \\
    Diatermano & sì & & \\
    Rigido & & no & \\
    Deformabile & & sì & \\
    Chiuso (impermeabile) & & & no \\
    Aperto (permeabile) & & & sì \\
    Isolato & no & no & no \\
    \bottomrule
\end{tabular}

\section{Principi della termodinamica}

\subsection{Principi di conservazione}
\begin{itemize}
    \item Conserazione della \textbf{massa};
    \item Conserazione della \textbf{energia} (I principio);
    \item Conserazione della \textbf{entropia} (II principio);
\end{itemize}

\subsection{Primo principio della termodinamica}

\[\Delta U = Q^\shortleftarrow - L^\shortrightarrow \]

\begin{tabular}{ll}
    $U$ & energia interna del sistema \\
    $Q^\shortleftarrow$ & calore entrante nel sistema \\
    $L^\shortrightarrow$ & lavoro uscente dal sistema \\
\end{tabular}

L'energia interna è una grandezza estensiva:
\[U_z = U_1 + U_2 + \ldots\]

Per una trasformazione ciclica:
\[\Delta U_{\text{ciclo}} = 0 \]

Per i sistemi conservativi:
\[\var{L} = \dd{E_c} + \dd{E_p}\]


\subsection{Secondo principio della termodinamica}

Per un sistema all'equilibrio esiste una funzione di stato detta \textbf{entropia} $S$ la cui variazione per una trasformazione \emph{reversibile} è data da:
\[\Delta S = \int \frac{\dd{Q_{rev}^\shortleftarrow}}{T}\]

In un sistema isolato in cui avvengono trasformazioni $\Delta S \ge 0$ e tende a zero al limite della reversibilità.

L'entropia è una grandezza estensiva:
\[
    S_z = S_1 + S_2 + \ldots \qquad \Delta S_z = \Delta S_1 + \Delta S_2 + \ldots
\]

In un sistema chiuso il bilancio entropico è dato da:
\[
    \Delta S = S_Q^\shortleftarrow + S_{irr} \qquad S_{irr} \ge 0 \qquad \text{segno}~S_Q^\shortleftarrow = \text{segno}~Q^\shortleftarrow
\]

\begin{description}
    \item[Enunciato di Clausius]Non esiste una macchina ciclica che non produca altro effetto che il trasferimento di calore da una sorgente fredda a una sorgente calda.

\item[Enunciato di Kelvin]
Non esiste una macchina ciclica il cui unico effetto sia l'assorbimento di calore da una sorgente calda e la produzione di un'equivalente quantità di lavoro.
\end{description}

\section{Lavoro termodinamico e calore}

\subsection{Lavoro termodinamico}

\begin{align*}
    \delta L^\sra &= PA\dd{s} = P\dd{V} \\
    \delta l^\sra &= P\dd{v} \\
    l^\sra &= \int_i^f P\dd{v}
\end{align*}

Il lavoro dipende dal percorso: \textbf{non} è funzione di stato.

\subsection{Calore}

\begin{align*}
    \text{Capacità termica} &\qquad C_x = \left(\frac{\delta Q^\sla}{\dd{T}}\right)_x \\
    \text{Calore specifico} &\qquad c_x = \frac{1}{M} \left(\frac{\delta Q^\sla}{\dd{T}}\right)_x
\end{align*}

\begin{align*}
    \text{A pressione cost.} &\qquad c_P = \frac{1}{M} \left(\frac{\delta Q^\sla}{\dd{T}}\right)_P = \left(\frac{\delta q^\sla}{\dd{T}}\right)_P \\
    \text{A volume cost.} &\qquad c_v = \frac{1}{M} \left(\frac{\delta Q^\sla}{\dd{T}}\right)_v = \left(\frac{\delta q^\sla}{\dd{T}}\right)_v
\end{align*}

Inoltre si ha che
\begin{align*}
    \text{A pressione cost.} &\qquad c_P = \qty(\pdv{h}{T})_P\\
    \text{A volume cost.} &\qquad c_v = \qty(\pdv{u}{T})_v
\end{align*}

\subsection{Entalpia}

L'entalpia è una funzione di stato.

\[ \dd{h} = \dd{u} + v\dd{P} + P\dd{v} = \delta q^\sla + v\dd{P} \]
\[ \delta q^\sla = \dd{h} - v\dd{P} \]

\subsection{Gas ideali ed energia interna}
Nei gas ideali, a seguito di evidenze sperimentali, si ha che $u = u(T)$, ovvero l'\textbf{energia interna dipende solo dalla temperatura}.
Quindi anche l'entalpia, $c_v$ e $c_P$ sono dipendenti solo dalla temperatura e si ha che
\begin{align*}
    \text{A pressione cost.} &\qquad c_P = \qty(\dv{h}{T})\\
    \text{A volume cost.} &\qquad c_v = \qty(\dv{u}{T})
\end{align*}

\subsection{Relazione di Mayer}
\[ c_P = \dv{h}{T} = \frac{\dd{u} + \dd{\qty(Pv)}}{\dd{T}} = \qty(\dv{u}{T}) + \dv{R^*T}{T} = c_v + R^* \]

Per i gas \textbf{perfetti} (gas ideali per variazioni di temperatura non troppo elevate) vale che:

{\renewcommand\arraystretch{1.4}
\begin{tabular}{lcc}
    \toprule
    & $c_v$ & $c_p$ \\ \midrule
    Gas Monoatomico & $\frac{3}{2}R^*$ & $\frac{5}{2}R^*$ \\
    Gas Biatomico o Poliatomico lineare & $\frac{5}{2}R^*$ & $\frac{7}{2}R^*$ \\
    Gas Poliatomico non lineare & $\frac{6}{2}R^*$ & $\frac{8}{2}R^*$ \\
    \bottomrule
\end{tabular}}

\begin{tabular}{ll}
    Monoatomico & \ch{He}, \ch{Ar}, $\ldots$ \\
    Linare & \ch{O2}, \ch{N2}, \ch{H2}, \ch{CO2}, $\ldots$ \\
    Non lineare & \ch{CH4}, \ch{H2O}, $\ldots$ \\
\end{tabular}

Liquido incomprimibile ideale $C_v = C_P = c(T)$.

Liquido incomprimibile perfetto $C_v = C_P = \text{cost}$.


\subsection{Politropiche}
Data una trasformazione quasi statica per un sistema con un \textbf{gas ideale} dove $c_x = \text{cost}$.

\[
    Pv^n = \text{cost} \quad Tv^{n-1} = \text{cost} \quad \frac{T^n}{P^{n-1}} = \text{cost} \quad PT^{\frac{n}{1-n}} = \text{cost}
\]

\begin{tabular}{lcc}
    \toprule
    Trasformazione & $c_x$ & $n = \frac{c_x-c_P}{c_x-c_v}$ \\
    \midrule
    Isoterma & $\pm\infty$ & 1 \\
    Isocora & $c_v$ & $\pm\infty$ \\
    Isobara & $c_P$ & 0 \\
    Adiabatica & 0 & $k = \frac{c_p}{c_v}$ \\
    \bottomrule
\end{tabular}

Per $n \ne 1$ (quindi non isoterma) l'integrale del lavoro diventa:
\begin{align*}
    l^\sra = \int_1^2 P\dd{v} &= \frac{P_1v_1}{n-1} \qty[1-\qty(\frac{v_1}{v_2})^{n-1}] \\
    &= \frac{P_1v_1}{n-1} \qty[1-\qty(\frac{P_2}{P_1})^{\frac{n-1}{n}}]
\end{align*}
Per $n = 1$ l'integrale del lavoro diventa:
\[ l^\sra = \int_1^2 P\dd{v} = P_1v_1\ln{\frac{v_2}{v_1}} = P_1v_1\ln{\frac{P_1}{P_2}} \]

\subsection{Diagramma T-s}

L'area sottesa dalla curva in una trasformazione \emph{internamente reversibile} è uguale al calore scambiato:
\[ \dd{S}_{rev} = \frac{\dd{Q}_{rev}}{T} \qquad Q_{rev} = \int_i^f \dd{Q}_{rev} = \int_i^f T(S) \dd{S} \]

Nel piano T-s tutte le trasformazioni politropiche sono esponenziali:
\[T = T_0 e ^ { \frac{s-s_0}{c_x} }\]

\begin{tabular}{lll}
    Isoterme & $c_x = \infty$ & rette orizzontali \\
    Adiabatiche reversibili & $c_x = 0$ & rette verticale \\
    Isocore & $c_v < c_P$ & più ripide delle isobare \\
\end{tabular}

\subsubsection{Per i gas perfetti}
\begin{tabular}{lll}
    \toprule
    Trasformazione & $l=\int P\dd{v}$ & $q = \int \dd{q}$ \\
    \midrule
    $P = \text{cost}$ & $P\Delta v$ & $c_p\Delta T$ \\
    $v = \text{cost}$ & 0 & $c_v\Delta T$ \\
    $T = \text{cost}$ & $R^*T\ln{\frac{v_2}{v_1}} =$ & $R^*T\ln{\frac{v_2}{v_1}} =$ \\
    & $-R^*T\ln{\frac{P_2}{P_1}}$ & $-R^*T\ln{\frac{P_2}{P_1}}$ \\
    $\dd{Q}=0$ & $-c_v\Delta T$ & 0 \\
    $c_x = \text{cost}$ & $(c_x-c_v)\Delta T$ & $c_x\Delta T$ \\
    \bottomrule
\end{tabular}

\subsubsection{Per i gas perfetti}
\[ \Delta u = c_v \Delta T \qquad \Delta h = c_p \Delta T \]
\begin{align*}
    \Delta s &= c_v \ln{\frac{T_2}{T_1}} + R^*\ln{\frac{v_2}{v_1}} \\
    &= c_p \ln{\frac{T_2}{T_1}} - R^*\ln{\frac{P_2}{P_1}} \\
    &= c_p \ln{\frac{v_2}{v_1}} + c_v\ln{\frac{P_2}{P_1}} \\
\end{align*}

\subsubsection{Per i liquidi incomprimibili}

Se perfetti:

\[ \Delta u = c \Delta T \qquad \Delta s = c \ln{\frac{T_2}{T_1}} \]

Se ideali $c_p = c(T)$, $\beta = 0$ e $K_T = 0$:
\[ \dd{h} = c(T)\dd{T} + v\dd{P} \qquad \dd{s} = c(T)\frac{\dd{T}}{T} \]

Se perfetti:
\[ \Delta h = c\Delta T + v \Delta P \]

\section{Macchina termodinamica}
La macchina termodinamica è un sistema termodinamico composto ed isolato che nel caso più semplice è realizzato da
\begin{itemize}
\item due serbatoi di calore
\item un serbatoio di lavoro
\item una macchina ciclica che è in grado di interagire con continuità con i serbatoi di calore e lavoro
\end{itemize}

\begin{description}
\item[Serbatoio di calore]
sistema termodinamico che scambia solo calore con l'esterno senza alterare il suo stato interno;
gli scambi avvengono con trasformazioni quasi-statiche internamente reversibili.
\item[Serbatoio di lavoro]
sistema termodinamico che scambia solo lavoro con l'esterno senza alterare il suo stato interno;
gli scambi avvengono con trasformazioni quasi-statiche internamente reversibili.
\end{description}

\subsubsection{Risoluzione problemi macchine termiche}
Bisogna impostare e risolvere il sistema contente le equazioni di bilancio
\[
    \begin{cases}
    \Delta U_Z = 0 \\
    \Delta S_Z = S_{irr} \\
    \end{cases}
\]

Per macchina motrice con masse infinite:
\[
    \begin{cases}
        -Q_C + Q_F + L = 0 \\
        -\frac{Q_C}{T_C} + \frac{Q_F}{T_F} = S_{irr} \\
    \end{cases}
\]

Per macchina operatrice con masse infinite:
\[
    \begin{cases}
        Q_C - Q_F - L = 0 \\
        \frac{Q_C}{T_C} - \frac{Q_F}{T_F} = S_{irr} \\
    \end{cases}
\]

\subsection{Rendimenti macchine termiche}
{\renewcommand{\arraystretch}{1.5}
\begin{tabular}{p{3.5cm}r}
\textbf{Macchina motrice} & $\eta = \frac{L}{Q_C} \quad \eta_{II} = \frac{\eta}{\eta_{rev}} = \frac{L}{L_{rev}} $ \\
→ serbatoi a massa infinita & $\eta = 1 - \frac{T_F}{T_C} - \frac{T_F}{Q_C}S_{irr}$ \\
\phantom{→}$\put(3,2){\line(0,1){5}}\to$ reversibile ($S_{irr} = 0$) & $\eta_{rev} = 1 - \frac{T_F}{T_C}$ \\
\\
\textbf{Macchina operatrice} & (serbatoi a temp. cost.) \\
→ frigorifera & $\varepsilon_f = \frac{Q_F}{L} \quad \eta_{II} = \frac{\varepsilon}{\varepsilon_{rev}} = \frac{L_{rev}}{L}$ \\
\phantom{→}$\put(3,2){\line(0,1){5}}\to$ reversibile & $\varepsilon_{f,rev} = \frac{T_F}{T_C - T_F}$ \\
→ pompa di calore & $\varepsilon_{pdc} = \frac{Q_C}{L}$ \\
\phantom{→}$\put(3,2){\line(0,1){5}}\to$ reversibile & $\varepsilon_{pdc,rev} = \frac{T_C}{T_C - T_F}$ \\
\end{tabular}
}

\section{Sistemi aperti}
\subsection{Bilancio di massa}
\[ \dv{m}{t} = \sum_i \dot{m}_i^\sla \]
\textbf{Equazione di continuità} \\
\[ \dot{m} = \rho w \Omega \]

\subsection{Bilancio di energia}
\[ \dv{E}{t} = \sum_i \dot{E}_i^\sla \]
dove $E$ rappresenta l'energia associata al trasporto di massa e al lavoro e calore scambiato

\textbf{Lavoro di pulsione} \\
\[ L_P^\sla = - \int^{V}_{V+V_m} P\dd{V} = PV_m = m_i P v_i \]
\textbf{Energia associata al trasporto di massa} \\
\[ E_m = \sum_i m_i^\sla \qty(u + gz + \frac{w^2}{2}) \]
\textbf{Bilancio energetico} \\
\[ \dv{E}{t} = \sum_i \dot{m}_i^\sla \qty(h + gz + \frac{w^2}{2}) + \dot{Q}^\sla - \dot{L_e}^\sra\]

\subsection{Bilancio di entropia}
\[ \dv{S}{t} = \sum_i \dot{m}_i^\sla s_i + \dot{S}_{Q^\sla} + \dot{S}_{irr} \]

\subsection{Regime stazionario}
\[ \dot{m}_{ingresso}^\sla = -\dot{m}_{uscita}^\sla \]

\[ \dot{m}^\sla\qty[(h_i - h_u) + g(z_i-z_u) + \frac{(w_i^2-w_u^2)}{2}] + \dot{Q}^\sla - \dot{L}_e^\sra = 0 \]

\[ \dot{m}^\sla (s_i - s_u) + \dot{S}_{Q^\sla} + \dot{S}_{irr} = 0 \]

\section{Macchina aperta}
\textbf{Turbina, compressore e pompa}
\[ \dot{m}^\sla (h_i - h_u) - \dot{L}_e^\sra = 0 \]
\[ \dot{m}^\sla (s_i - s_u) + \dot{S}_{irr} = 0 \]

\textbf{Scambiatore di calore}
\[ \dot{m}^\sla (h_i - h_u) + \dot{Q}^\sla = 0 \]
\[ \dot{m}^\sla (s_i - s_u) + \dot{S}_{Q^\sla} + \dot{S}_{irr} = 0 \]

\textbf{Diffusore (w↓) e ugello (w↑)}
\[  (h_i - h_u) + \frac{(w_i^2 - w_u^2)}{2} = 0 \]
\[ \dot{m}^\sla (s_i - s_u) + \dot{S}_{irr} = 0 \]

\textbf{Valvola di laminazione}
\[  (h_i - h_u) = 0 \]
\[ \dot{m}^\sla (s_i - s_u) + \dot{S}_{irr} = 0 \]

\subsection{Rendimento isoentropico}
\textbf{Turbina}
\[ \eta_{is,turbina} = \frac{\dot{L}_{reale}^\sra}{\dot{L}_{ideale}^\sra} = \frac{h_1 - h_{2,reale}}{h_1 - h_{2,ideale}} \]

\textbf{Compressore}
\[ \eta_{is,compressore} = \frac{\dot{L}_{ideale}^\sra}{\dot{L}_{reale}^\sra} = \frac{h_1 - h_{2,ideale}}{h_1 - h_{2,reale}} \]

\subsection{Lavoro specifico per unità di massa fluente}
Dopo vari passaggi si ottiene che
\[ -\delta l_e^\sra = v\dd{P} + T\dd{s_{irr}} \]
Integrando fra la sezione di ingresso e quella di uscita:
\[ l_e^\sra = -\int_i^u v\dd{P} -\int_i^u T\dd{s_{irr}} \]
dove il secondo termine è l'energia dissipata per irreversibilità interna.

\section{Sistemi bifase}

Le grandezze estensive specifiche sono la media pesata sulle masse:
\[m = m_\alpha + m_\beta \qquad E = E_\alpha + E_\beta\]
\[e = \frac{m_\alpha}{m}e_\alpha + \frac{m_\beta}{m}e_\beta\]
\[\text{frazione massica:} \quad x_\alpha = \frac{m_\alpha}{m} \quad x_\beta = \frac{m_\beta}{m} \]

Dalla regola di Gibbs il numero di variabili intensive indipendenti per bifase monocomponente è 1, pressione e temperatura non sono indipendenti.

La transizione di fase è a P costante: $\dd{h} = \delta q$.

\subsection{Stati di aggregazione}
\begin{tabular}{p{2.7cm}p{4.5cm}}
    Liq. sottoraffreddato & Non in procinto di evaporare \\
    Liq. saturo & In procinto di evaporare \\
    Vapore umido & Stato di transizione \\
    Vapore saturo & In procinto di condensazione \\
    Vapore surriscaldato & Non in procinto di condensazione \\
\end{tabular}

\subsection{Entalpia di transizione}

\[ h_\text{solido} < h_\text{liquido} < h_\text{vapore} \]

Per l'acqua allo stato triplo:
\begin{tabular}{ll}
    Solidificazione & $h_{lst,\ch{H2O}} = \SI{-333}{kJ/Kg}$ \\
    Evaporazione & $h_{lvt,\ch{H2O}} = \SI{2501.6}{kJ/Kg}$
\end{tabular}

\subsection{Titolo di vapore, liquido e solido}
\[ x = x_v = \frac{m_v}{m} \quad x_l = \frac{m_l}{m} \quad x_s = \frac{m_s}{m} \]
\[ x_v + x_l + x_s = 1 \]

Per interpolare nelle tabelle:
\[ \frac{X-X_1}{X_2-X_1} = \frac{T-T_1}{T_2-T_1} \]
\[ Y = Y_1 + \frac{Y_2-Y_1}{X_2-X_1}(X-X_1) \]

\subsection{Approssimazioni per entropia e entalpie di solidi e liquidi}

Formule valide per l'acqua, partendo da uno stato di riferimento, per esempio il punto triplo.

Allo stato solido:
\[ h = h_0 + h_{lst} + c_s(T-T_0) + v(P-P_0) \]
\[ s = s_0 + s_{lst} + c_s\ln{\frac{T}{T_0}} = s_0 + \frac{h_{lst}}{T_0} + c_s\ln{\frac{T}{T_0}} \]

Allo stato liquido:
\[ h = h_0 + c_l(T-T_0) + v(P-P_0) \]
\[ s = s_0 + c_l\ln{\frac{T}{T_0}} \]

Con le tabelle è possibile trovare $h$ per l'acqua sottoraffreddata:
\[ h = h_{ls}(P_{sat}(T)) + v(P-P_{sat}(T)) \]

dove $v = v_{ls}(P_{sat}(T))$. Di solito il termine $v\Delta P$ è trascurabile.

\section{Cicli simmetrici}

Un ciclo formato da 4 politropiche, uguali a due a due.

\begin{minipage}{.6\linewidth}
    \begin{tikzpicture}[
        thick,
        >=stealth',
        dot/.style = {
        draw,
        fill = white,
        circle,
        inner sep = 0pt,
        minimum size = 4pt
        }
    ]
    \coordinate (O) at (0,0);
    \draw[->] (0,0) -- (4,0) coordinate[label = {below:$v$}] (xmax);
    \draw[->] (0,0) -- (0,4) coordinate[label = {left:$P$}] (ymax);

    \draw (3.5,0.5) node[below right] {4} parabola (1.5,1) node[below left] {1};
    \draw (1.5,1) parabola (0.5,3.5) node[above left] {2};
    \draw (2.5,3) parabola (0.5,3.5);
    \draw (3.5,0.5) parabola (2.5,3) node[above right] {3};

    \draw (1.5,3.2) node[above] {$m$};
    \draw (2,0.8) node[below] {$m$};
    \draw (0.8,2) node[left] {$n$};
    \draw (3.3,2) node[left] {$n$};
    \end{tikzpicture}
\end{minipage}%
\begin{minipage}{.35\linewidth}
    \begin{align*}
        v_1v_3 &= v_2v_4 \\
        P_1P_3 &= P_2P_4 \\
        T_1T_3 &= T_2T_4 \\
        \\
        P_1v_1^n &= P_2v_2^n \\
        P_2v_2^m &= P_3v_3^m \\
        P_3v_3^n &= P_4v_4^n \\
        P_4v_4^m &= P_1v_1^m \\
    \end{align*}
\end{minipage}

\section{Ciclo Carnot}
Il ciclo di Carnot è un ciclo simmetrico composto da due isoentropiche (adiabatiche reversibili) e due isoterme.

Il rendimento ciclo di Carnot vale:
\[ \eta = \frac{L}{Q_C} = 1 - \frac{Q_F}{Q_C} = 1 - \frac{T_1}{T_3} \]

\begin{minipage}{.5\linewidth}
\begin{tikzpicture}[thick,>=stealth']
    \coordinate (O) at (0,0);
    \draw[->] (0,0) -- (3.5,0) coordinate[label = {below:$s$}] (xmax);
    \draw[->] (0,0) -- (0,3.5) coordinate[label = {left:$T$}] (ymax);
    
    \draw (1,1) node[below left] {1} -- (1,2.5) node[above left] {2} parabola (3,2.5) node[above right] {3} -- (3,1) node[below] {4};
    \draw (1,1) parabola (3,1);
\end{tikzpicture}
\end{minipage}%
\begin{minipage}{.5\linewidth}
\begin{tikzpicture}[thick,>=stealth']
    \coordinate (O) at (0,0);
    \draw[->] (0,0) -- (3.5,0) coordinate[label = {below:$v$}] (xmax);
    \draw[->] (0,0) -- (0,3.5) coordinate[label = {left:$P$}] (ymax);

    \draw (3,1) node[below] {4} parabola (1.25,1.35) node[below left] {1};
    \draw (1.25,1.35) parabola (0.5,3) node[above left] {2};
    \draw (2.25,2.25) parabola (0.5,3);
    \draw (3,1) parabola (2.25,2.25) node[above right] {3};
\end{tikzpicture}
\end{minipage}

\subsection{Irreversibilità}
Se \textbf{irreversibilità esterna} ($T_1 > T_F$ e $T_2 < T_C$):
\[ \eta_{rev} = 1 - \frac{T_F}{T_C} > \eta_{ciclo} = 1 - \frac{T_1}{T_3} \]
Bilancio entropico su tutta macchina termica:
\[ -\frac{Q_C}{T_C} + \frac{Q_F}{T_F} = S_{irr} \]
Bilancio entropico su macchina ciclica (ciclo di Carnot):
\[ \frac{Q_C}{T_3} = \frac{Q_F}{T_1} = \Delta S \]
Da cui si ricava:
\[ Q_C \qty(\frac{T_1}{T_F T_3} - \frac{1}{T_C}) = S_{irr} \]

Se \textbf{irreversibilità interna} ($s_1 < s_2$ e $s_3 < s_4$):\\
Bilancio entropico su tutta macchina termica:
\[ -\frac{Q_C}{T_C} + \frac{Q_F}{T_F} = S_{irr} \]
Bilancio entropico su macchina ciclica:
\[ \frac{Q_C}{T_C} = S_3 - S_2 \qq{e} \frac{Q_F}{T_F} = S_4 - S_1 \]
Da cui deriva: $S_2 - S_3 + S_4 - S_1 = S_{irr} > 0$

\section{Ciclo Joule-Brayton}

Ciclo simmetrico costituito da due \emph{isoentropiche} e due \emph{isobare}.
Applicazioni: impianti a turbina a gas con ciclo chiuso o aperto.

\begin{minipage}{.5\linewidth}
\begin{tikzpicture}[thick,>=stealth']
    \coordinate (O) at (0,0);
    \draw[->] (0,0) -- (3.5,0) coordinate[label = {below:$s$}] (xmax);
    \draw[->] (0,0) -- (0,3.5) coordinate[label = {left:$T$}] (ymax);
    
    \draw (1,1) node[below left] {1} -- (1,1.7) node[above left] {2} parabola (3,3) node[above right] {3} -- (3,1.5) node[below] {4};
    \draw (1,1) parabola (3,1.5);
\end{tikzpicture}
\end{minipage}%
\begin{minipage}{.5\linewidth}
\begin{tikzpicture}[thick,>=stealth']
    \coordinate (O) at (0,0);
    \draw[->] (0,0) -- (3.5,0) coordinate[label = {below:$v$}] (xmax);
    \draw[->] (0,0) -- (0,3.5) coordinate[label = {left:$P$}] (ymax);

    \draw (3.5,1) node[below] {4} -- (1.5,1) node[below left] {1};
    \draw (1.5,1) parabola (0.5,3) node[above left] {2};
    \draw (2.5,3) parabola (0.5,3);
    \draw (3.5,1) parabola (2.5,3) node[above right] {3};
\end{tikzpicture}
\end{minipage}

Nell'ipotesi di \emph{gas perfetto} e \emph{ciclo ideale simmetrico}:
\begin{align*}
    \dot{Q}_c &= \dot{m} (h_3-h_2) & \qquad \dot{Q}_f &= \dot{m} (h_4 - h_1) \\
    \dot{Q}_c &= \dot{m} c_p (T_3 - T_2) & \qquad \dot{Q}_f &= \dot{m} c_p (T_4 - T_1) \\
\end{align*}

Il rendimento termodinamico del ciclo vale:
\[
    \eta_{JB} = \frac{\dot{L}}{\dot{Q}_c} = 1 - \frac{\dot{Q}_f}{\dot{Q}_c} = 1 - \frac{T_4-T_1}{T_3-T_2} = 1 - \frac{T_1}{T_2}
\]

Definendo il rapporto di compressione: $r_p = \frac{P_2}{P_1}$

\[
    \eta_{JB} = 1 - \frac{1}{r_p^{\frac{k-1}{k}}}
\]

\begin{align*}
    \min \eta ~\text{per}~ r_p = 1 \quad \max \eta ~\text{per}~ T_2 \sra T_3 \Rightarrow r_{p}^{\max{}} = \left(\frac{T_3}{T_1}\right)^\frac{k}{k-1}
\end{align*}

\subsection{Lavoro specifico per ciclo Joule-Brayton}
\[
    l = l_T - l_c = c_p (T_3 - T_4) - c_p (T_2 - T_1)
\]
Il lavoro massimo si ha in corrispondenza di:
\[
    r_p^{\text{opt}} = \left(\frac{T_3}{T_1}\right)^\frac{k}{2(k-1)} = \sqrt{r_p^{\max{}}} \qquad\text{e}\qquad T_4 = T_2 = \sqrt{T_1T_3}
\]

\section{Ciclo Joule-Brayton con rigenerazione}

\begin{minipage}{.5\linewidth}
\begin{tikzpicture}[thick,>=stealth']
    \coordinate (O) at (0,0);
    \draw[->] (0,0) -- (3.5,0) coordinate[label = {below:$s$}] (xmax);
    \draw[->] (0,0) -- (0,3.5) coordinate[label = {left:$T$}] (ymax);
    
    \draw (1,1) node[below left] {1} -- (1,1.7) node[above left] {2} parabola (3,3) node[above right] {3} -- (3,2) node[right] {4};
    \draw (1,1) parabola (3,2);
    \draw[dashed] (1,1.7) -- (2.68,1.7) node[right] {4'};
    \draw[dashed] (1.95,2) node[above left] {2'} -- (3,2);
\end{tikzpicture}
\end{minipage}%
\begin{minipage}{.5\linewidth}
    Si può applicare se $T_2 \le T_4$
\end{minipage}

Il rendimento nel caso $T_2 = T_4$ è:
\begin{align*}
    \eta_{\text{rig}} &= \frac{\dot{L}}{\dot{Q}_c} = \frac{\dot{L}_T-\dot{L}_c}{\dot{Q}_c} = \frac{ (T_3-T_4) - (T_2-T_1) }{T_3-T_4} = \\
        &= 1 - \frac{T_2-T_1}{T_3-T_4} = 1 - \frac{T_2}{T_3} = 1 - \frac{T_2T_1}{T_3T_1} = 1 - \frac{T_1}{T_3}r_p^{\frac{k-1}{k}}
\end{align*}

\section{Ciclo Otto}
Ciclo simmetrico cosituito da due \emph{isoentropiche} e due \emph{isocore}.
Applicazioni: prevalentemente in campo automobilistico. $r_v$ tra 6 e 10 per evitare autocombustione della miscela in $1-2$.

\begin{minipage}{.5\linewidth}
    \begin{tikzpicture}[thick,>=stealth']
        \coordinate (O) at (0,0);
        \draw[->] (0,0) -- (3.5,0) coordinate[label = {below:$s$}] (xmax);
        \draw[->] (0,0) -- (0,3.5) coordinate[label = {left:$T$}] (ymax);
        
        \draw (1,1) node[below left] {1} -- (1,1.3) node[above left] {2} parabola (3,3) node[above right] {3} -- (3,2) node[below] {4};
        \draw (1,1) parabola (3,2);
    \end{tikzpicture}
\end{minipage}%
\begin{minipage}{.5\linewidth}
    \begin{tikzpicture}[thick,>=stealth']
        \coordinate (O) at (0,0);
        \draw[->] (0,0) -- (3.5,0) coordinate[label = {below:$v$}] (xmax);
        \draw[->] (0,0) -- (0,3.5) coordinate[label = {left:$P$}] (ymax);
    
        \draw (3.2,0.5) node[right] {1} parabola (0.7,1) node[left] {2} -- (0.7,3) node[left] {3} parabola bend (3.2,1) (3.2,1) node[right] {4} -- cycle;
    \end{tikzpicture}
\end{minipage}

Il rendimento del ciclo Otto nel caso di \emph{gas perfetto}:
\[
    \eta_{\text{Otto}} = 1 - \frac{q_f}{q_c} = 1 - \frac{T_4-T_1}{T_3-T_2} = 1 - \frac{T_1}{T_2}
\]

Dal bilancio entropico tra 1 e 2 vale:
\[
    \left( \frac{T_2}{T_1} \right)^{c_v} = \left( \frac{V_1}{V_2} \right)^{R^*} 
\]
Chiamando rapporto di compressione volumentrico $r_v = \frac{V_1}{V_2}$
\[
    \eta_{\text{Otto}} = 1 - r_v^{1-k}
\]

\subsection{Lavoro specifico per il ciclo Otto}
\[
    l = c_v(T_3-T_4) - c_v(T_2-T_1) = c_vT_3\left(1-\frac{1}{r_v^{k-1}}\right) - c_vT_1(r_v^{k-1}-1)
\]
\[
    r_v^{\text{opt}} = \left(\frac{T_3}{T_1}\right)^{\frac{1}{2(k-1)}}
\]

\section{Ciclo Diesel}

Ciclo \textbf{NON simmetrico} costituito da due \emph{isoentropiche}, una \emph{isobara} e una \emph{isocora}.
Applicazioni: motore endotermico.

\begin{minipage}{.5\linewidth}
\begin{tikzpicture}[thick,>=stealth']
    \coordinate (O) at (0,0);
    \draw[->] (0,0) -- (3.5,0) coordinate[label = {below:$s$}] (xmax);
    \draw[->] (0,0) -- (0,3.5) coordinate[label = {left:$T$}] (ymax);
    
    \draw (1,0.5) node[below left] {1} -- (1,1.25) node[above left] {2} parabola (3,3) node[above right] {3} -- (3,1.5) node[below] {4};
    \draw (1,0.5) parabola (3,1.5);
\end{tikzpicture}
\end{minipage}%
\begin{minipage}{.5\linewidth}
\begin{tikzpicture}[thick,>=stealth']
    \coordinate (O) at (0,0);
    \draw[->] (0,0) -- (3.5,0) coordinate[label = {below:$v$}] (xmax);
    \draw[->] (0,0) -- (0,3.5) coordinate[label = {left:$P$}] (ymax);

    \draw (2.75,0.75) node[below right] {1} [looseness=1.25,out=-180,in=-90] to (0.5,3) node[above left] {2} -- (1.25,3) node[above right] {3};
    \draw (2.75,1.5) node[right] {4} [looseness=1.25,out=-180,in=-90] to (1.25,3);
    \draw (2.75,0.75) -- (2.75,1.5);
\end{tikzpicture}
\end{minipage}

Nell'ipotesi di \emph{gas perfetto} e \emph{ciclo ideale}:
\[ \eta_{diesel} = \frac{L}{Q_C} = 1 - \frac{c_v (T_4-T_1)}{c_p (T_3-T_2)} = 1 - \frac{c_v T_1 (\frac{T_4}{T_1}-1)}{c_p T_2 (\frac{T_3}{T_2}-1)} \]

\begin{align*}
\text{Rapporto compressione volumetrico:} &\qquad r = \frac{V_1}{V_2} \\
\text{Rapporto di combustione:} &\qquad z = \frac{V_3}{V_2} \\
\end{align*}

\[ \eta_{diesel} = 1 - \frac{1}{r^{k-1}} \frac{1}{k} \frac{z^k-1}{z-1} \]

\subsection{Confronto ciclo Otto - ciclo Diesel}
\begin{itemize}
    \item Motore Otto è più leggero a parità di potenza
    \item Motore Otto ha frequenza di rotazione maggiore
    \item Motore Otto ha minore rumorosità
    \item Motore Diesel ha miglior rendimento a causa del maggior rapporto di compressione (circa il doppio rispetto a motore Otto)
    \item Motore Diesel ha miglior rendimento al diminuire del carico (più facilmente regolabile in potenza)
    \item Motore Diesel utilizza un combustibile meno pregiato del motore Otto
\end{itemize}

\section{Ciclo Joule-Brayton inverso}

Ciclo frigorifero inverso costituito da due \emph{isoentropiche} e due \emph{isobare}.

\begin{minipage}{.5\linewidth}
    \begin{tikzpicture}[thick,>=stealth']
        \coordinate (O) at (0,0);
        \draw[->] (0,0) -- (3.5,0) coordinate[label = {below:$s$}] (xmax);
        \draw[->] (0,0) -- (0,3.5) coordinate[label = {left:$T$}] (ymax);

        \draw (1,1) node[below left] {1} -- (1,1.7) node[above left] {4} parabola (3,3) node[above right] {3} -- (3,1.5) node[below] {2};
        \draw (1,1) parabola (3,1.5);
    \end{tikzpicture}
\end{minipage}%
\begin{minipage}{.5\linewidth}
    \begin{tikzpicture}[thick,>=stealth']
        \coordinate (O) at (0,0);
        \draw[->] (0,0) -- (3.5,0) coordinate[label = {below:$v$}] (xmax);
        \draw[->] (0,0) -- (0,3.5) coordinate[label = {left:$P$}] (ymax);

        \draw (3.5,1) node[below] {2} -- (1.5,1) node[below left] {1};
        \draw (1.5,1) parabola (0.5,3) node[above left] {4};
        \draw (2.5,3) parabola (0.5,3);
        \draw (3.5,1) parabola (2.5,3) node[above right] {3};
    \end{tikzpicture}
\end{minipage}

Se il ciclo è simmetrico:
\[
    \varepsilon = \frac{\dot{Q}_f}{\dot{Q}_c-\dot{Q}_f} = \frac{T_2}{T_3-T_2} = \frac{T_1}{T_4-T_1} = \frac{1}{ r_p^{\frac{k-1}{k}} - 1 }
\]
Alta efficienza per $r_p \sra 1$.

\section{Altri cicli}

\begin{itemize}
    \item Ciclo Stirling: due \emph{isoterme} e due \emph{isocore};
    \item Ciclo Ericsson: due \emph{isoterme} e due \emph{isobare};
\end{itemize}

\section{Ciclo Carnot a vapore}
\begin{tikzpicture}[thick,>=stealth']
    \coordinate (O) at (0,0);
    \draw[->] (0,0) -- (3.5,0) coordinate[label = {below:$s$}] (xmax);
    \draw[->] (0,0) -- (0,3.5) coordinate[label = {left:$T$}] (ymax);

    \gaussiana{-1.2}{1.2}{0}{0.7}

    \draw (1,1) node[below left] {1} -- (1,1.8) node[above left] {2} -- (2.42,1.8) node [above right] {3} -- (2.42,1) node [below right] {4} -- cycle;

\end{tikzpicture}

Vantaggi:
\begin{itemize}
    \item L'isobara nel bifase è anche isoterma (meno irreversibilità);
    \item La transizione di fase aumenta l'energia specifica scambiata lungo le isoterme;
\end{itemize}

Svantaggi:
\begin{itemize}
    \item Compressione $1-2$ di un bifase difficile da realizzare e soggetta a molte irreversibilità;
    \item L'espansione $3-4$ conviene se $x_4<0.9$.
\end{itemize}

\section{Fluidi di lavoro in cicli a vapore}

\begin{itemize}
    \item Elevata massa volumica e entalpia di transizione di fase $\Rightarrow$ ridurre la portata di fluido;
    \item Elevata temperatura critica $\Rightarrow$ al punto critico $h_{lvt} = 0$;
    \item Temperatura del punto triplo inferiore a quella minima del ciclo $\Rightarrow$ evitare fase solida;
    \item Fluido non corrosivo $\Rightarrow$ ridurre i costi e manutenzione;
    \item Fluido non tossico $\Rightarrow$ ridurre i rischi ambientali;
    \item Chimicamente stabile $\Rightarrow$ aumentare sicurezza impianto;
    \item Facilmente reperibile e di basso costo;
    \item Elevata pendenza della curva limite superiore in $T-s$ $\Rightarrow$ vapore in uscita con elevato titolo;
    \item Pressione di condensazione $> \SI{1}{atm}$ $\Rightarrow$ evitare infiltrazioni di gas incondensabili;
\end{itemize}

Esempi di fluidi adatti:

\begin{itemize}
    \item Ciclo motore: acqua
    \item Ciclo frigorifero:
    \begin{itemize}
        \item Ammoniaca $\ch{NH3}$ $\Rightarrow$ tossico;
        \item Clorofluorocarburi ($\ch{CFC}$ o freon) $\Rightarrow$ dannosi per l'ozono;
        \item Clorofluoroidrocarburi ($\ch{HCFC}$) o fluoroidrocarburi ($\ch{HFC}$) $\Rightarrow$ meno dannosi per l'ozono ($R134a$);
    \end{itemize}
\end{itemize}

\section{Ciclo Rankine a vapore saturo}
\begin{tikzpicture}[thick,>=stealth']
    \coordinate (O) at (0,0);
    \draw[->] (0,0) -- (3.5,0) coordinate[label = {below:$s$}] (xmax);
    \draw[->] (0,0) -- (0,3.5) coordinate[label = {left:$T$}] (ymax);

    \gaussiana{-1.2}{1.2}{0}{0.7}

    \draw (0.54,1) node[below] {1} -- (0.54,1.3) node[left]{2} -- (0.98,1.8) node[above left] {3} -- (2.42,1.8) node [above right] {4} -- (2.42,1) node [below right] {5} -- cycle;

\end{tikzpicture}
\begin{tabular}{p{2.5cm}l}
Trasformazione 1-2: & compressione isoentropica (pompa)\\
Trasformazione 2-4: & riscaldamento isobaro (bruciatore)\\
Trasformazione 4-5: & espansione isoentropica (turbina)\\
Trasformazione 5-1: & raffreddamento isobaro (condensatore)\\
\end{tabular}

Rendimento ciclo Rankine a vapore saturo:
\[ \eta = 1 - \frac{\dot{Q}_F}{\dot{Q}_C} = 1 - \frac{\dot{m}(h_5-h_1)}{\dot{m}(h_4-h_2)} = 1 - \frac{h_5-h_1}{h_4-h_2} \]

Per il calcolo del salto entalpico in pompa essendo $\dd{h} = T\dd{s} + v\dd{P}$ ma $\dd{s} = 0$ e fluido incomprimibile:
\[ h_2-h_1 = v(P_2-P_1) \]

Il ciclo Rankine a vapore saturo non viene utilizzato perchè all'uscita dalla turbina il titolo non è sufficientemente alto (si vorrebbe avere titolo superiore a 0.9).

\section{Ciclo Rankine con surriscaldamento}
\begin{tikzpicture}[thick,>=stealth']
    \coordinate (O) at (0,0);
    \draw[->] (0,0) -- (3.5,0) coordinate[label = {below:$s$}] (xmax);
    \draw[->] (0,0) -- (0,3.5) coordinate[label = {left:$T$}] (ymax);

    \gaussiana{-1.2}{1.2}{0}{0.7}

    \draw (0.54,1) node[below] {1} -- (0.54,1.3) node[left]{2} -- (0.98,1.8) node[above left] {3} -- (2.42,1.8) node [below left] {4} parabola (2.8,3) node[right]{5} -- (2.8,1) node [below] {6} -- cycle;

\end{tikzpicture}
\begin{tabular}{p{2.5cm}l}
Trasformazione 1-2: & compressione isoentropica (pompa)\\
Trasformazione 2-5: & riscaldamento isobaro (bruciatore)\\
Trasformazione 5-6: & espansione isoentropica (turbina)\\
Trasformazione 6-1: & raffreddamento isobaro (condensatore)\\
\end{tabular}

Rendimento ciclo Rankine:
\[ \eta = 1 - \frac{\dot{Q}_F}{\dot{Q}_C} = 1 - \frac{\dot{m}(h_6-h_1)}{\dot{m}(h_5-h_2)} = 1 - \frac{h_6-h_1}{h_5-h_2} \]

\section{Soluzioni per aumento rendimento ciclo Rankine}

\begin{itemize}
    \item Abbassamento della pressione di condensazione $\Rightarrow$ maggior lavoro prodotto (ma anche più calore richiesto all'uscita della pompa, titolo minore in uscita dalla turbina e rischio di infiltrazioni se pressione di condensazione minore pressione atmosferica)
    \item Aumento della temperatura finale di surriscaldamento $\Rightarrow$ aumento di lavoro prodotto (ma anche aumento calore richiesto) e titolo in uscita dalla turbina più alto.
    \item Aumento della pressione di vaporizzazione (a parità di temperatura massima) $\Rightarrow$ stesso lavoro e meno calore in uscita durante condensazione (ma anche titolo minore in uscita da turbina)
    \item Surriscaldamenti ripetuti con espansioni in più stadi $\Rightarrow$ permettono di aumentare lavoro e titolo in uscita da turbina
    \item Rigenerazione $\Rightarrow$ si estrae vapore dalla turbina e si mette a contatto con il liquido a bassa temperatura
    \item Cogenerazione $\Rightarrow$ si utilizza vapore in uscita da turbina per teleriscaldamento o per altri scopi
\end{itemize}

\section{Ciclo Carnot inverso a vapore}
\begin{tikzpicture}[thick,>=stealth']
    \coordinate (O) at (0,0);
    \draw[->] (0,0) -- (3.5,0) coordinate[label = {below:$s$}] (xmax);
    \draw[->] (0,0) -- (0,3.5) coordinate[label = {left:$T$}] (ymax);

    \gaussiana{-1.2}{1.2}{0}{0.7}

    \draw (1,1) node[below left] {2} -- (1,1.8) node[above left] {1} -- (2.42,1.8) node [above right] {4} -- (2.42,1) node [below right] {3} -- cycle;
\end{tikzpicture}

\begin{itemize}
    \item L'espansione $1-2$ non è problematica;
    \item La compressione $3-4$ può condurre ad una elevata erosione del compressore;
\end{itemize}

\section{Ciclo frigorifero a vapore}
\begin{minipage}{.5\linewidth}
\begin{tikzpicture}[thick,>=stealth']
    \coordinate (O) at (0,0);
    \draw[->] (0,0) -- (3.5,0) coordinate[label = {below:$s$}] (xmax);
    \draw[->] (0,0) -- (0,3.5) coordinate[label = {left:$T$}] (ymax);

    \gaussiana{-1.2}{1.2}{0}{0.7}

    \draw (0.98,1.8) node[above left] {1} -- (2.42,1.8) node [below] {5} parabola (2.9,3) node[right] {4} -- (2.9,1) node [below] {3} -- (1.2,1);
    \draw[dotted] (1.2,1) node[below] {2} parabola (0.98,1.8);
\end{tikzpicture}
\end{minipage}%
\begin{minipage}{.5\linewidth}
\begin{tikzpicture}[thick,>=stealth']
    \coordinate (O) at (0,0);
    \draw[->] (0,0) -- (3.5,0) coordinate[label = {below:$h$}] (xmax);
    \draw[->] (0,0) -- (0,3.5) coordinate[label = {left:$P$}] (ymax);

    \draw [thin] plot [smooth, tension=1] coordinates {(0.6,0) (1,1.7) (2,2.6) (2.8,1.7) (2.2,0)};
    \draw (1.15,2) node[above left] {1} -- (2.9,2) node[above] {5} -- (3.5,2) node[above] {4} -- (2.7,1) node[right] {3} -- (1.15,1) node[below] {2};
    \draw [dotted] (1.15,1) -- (1.15,2);
\end{tikzpicture}
\end{minipage}

L'espansione $1-2$ isoentropica è sostituita da un'espansione adiabatica isoentalpica (valvola di laminazione).


%%%%%%%%%%%%%%%%%%%%%%%%%%%%%%%%%%%%%%%%%%%%%%%%%%%%%%%%%%%%%%%%%%%%%%%%%%%%%%%%
\vfill
\rule{\linewidth}{0.25pt}
\scriptsize\\
\href{mailto:marco.donadoni@mail.polimi.it}{M. Donadoni}, \href{mailto:edoardo.morassutto@mail.polimi.it}{E. Morassutto}, Politecnico di Milano, A.A. 2018/19
\end{multicols}
\end{document}
