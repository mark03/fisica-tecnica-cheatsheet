\section{Lavoro termodinamico e calore}

\subsection{Lavoro termodinamico}

\begin{align*}
    \delta L^\shortrightarrow &= PA\dd{s} = P\dd{V} \\
    \delta l^\shortrightarrow &= P\dd{v} \\
    l^\shortrightarrow &= \int_i^f P\dd{v}
\end{align*}

Il lavoro dipende dal percorso: \textbf{non} è funzione di stato.

\subsection{Calore}

\begin{align*}
    \text{Capacità termica} &\qquad C_x = \left(\frac{\delta Q^\shortleftarrow}{\dd{T}}\right)_x \\
    \text{Calore specifico} &\qquad c_x = \frac{1}{M} \left(\frac{\delta Q^\shortleftarrow}{\dd{T}}\right)_x
\end{align*}

\begin{align*}
    \text{A pressione cost.} &\qquad c_P = \frac{1}{M} \left(\frac{\delta Q^\shortleftarrow}{\dd{T}}\right)_P = \left(\frac{\delta q^\shortleftarrow}{\dd{T}}\right)_P \\
    \text{A volume cost.} &\qquad c_v = \frac{1}{M} \left(\frac{\delta Q^\shortleftarrow}{\dd{T}}\right)_v = \left(\frac{\delta q^\shortleftarrow}{\dd{T}}\right)_v
\end{align*}

Inoltre si ha che
\begin{align*}
    \text{A pressione cost.} &\qquad c_P = \qty(\pdv{h}{T})_P\\
    \text{A volume cost.} &\qquad c_v = \qty(\pdv{u}{T})_v
\end{align*}

\subsection{Entalpia}

L'entalpia è una funzione di stato.

\[ \dd{h} = \dd{u} + v\dd{P} + P\dd{v} = \delta q^\shortleftarrow + v\dd{P} \]
\[ \delta q^\shortleftarrow = \dd{h} - v\dd{P} \]
