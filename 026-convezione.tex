\section{Convezione}
Trasferimento di energia tra una superficie solida e un fluido adiacente in movimento.
\begin{description}
 \item[Convezione forzata] Avviene quando il fluido è forzato a scorrere su una superficie da mezzi esterni
 \item[Convezione naturale] Avviene quando il moto del fluido è causato da forze ascensionali che sono indotte dalle differenze di densità dovute alla variazione di temperatura del fluido in un campo gravitazionale
\end{description}

\subsection{Equazione generale}
\[ \dot{q} = h \cdot \qty(T_s - T_f) \]
\begin{align*}
h & \quad \text{coefficiente convettivo} \\
T_s & \quad \text{temperatura superficie} \\
T_f & \quad \text{temperatura fluido} \\
\end{align*}
\[ \qq*{Ponendo} R_{conv} = \frac{1}{h \cdot A} \qq{si ha che} \dot{Q}_{conv} = -\frac{\Delta T}{R_{conv}} \]

\subsection{Raggio critico di isolamento}
Considerando gusci cilindrici aventi come utimo strato uno strato di isolante soggetto a fenomeni convettivi.
\[ R_{tot} = R_{isol} + R_{conv} = \frac{\ln(\frac{R_e}{R_i})}{2\pi L k} + \frac{1}{2\pi R_e L h} \]
\begin{align*}
R_e & \quad \text{raggio esterno isolante} \\
R_i & \quad \text{raggio interno isolante}
\end{align*}
Si ha che inizialmente $R_{tot}$ decresce all'aumentare dello spessore di isolante.
Si ha $R_{tot}$ minimo quando $R_e = R_{critico} = \frac{k}{h}$.

\subsection{Coefficiente convettivo}

\begin{tabular}{p{5cm}r}
    \toprule
    Fluido & $h \quad \qty[\frac{W}{m^2K}]$ \\ \midrule
    gas stagnante &  \numrange{5}{50} \\
    acqua stagnante & $100$ \\
    gas in moto & \numrange{50}{1000} \\
    olio minerale & \numrange{50}{3000} \\
    acqua in moto & \numrange{200}{10000} \\
    acqua ebollizione/condensazione & \numrange{1000}{100000} \\
    metalli liquidi & \numrange{10000}{100000} \\
    \bottomrule
\end{tabular}\\

Il coefficiente convettivo non è una proprietà della materia e dipende da:
\begin{align*}
    \rho & \quad \text{densità} \\
    \mu & \quad \text{viscosità} \\
    c_p & \quad \text{cal. spec. pressione cost.} \\
    k & \quad \text{conduttività} \\
    w & \quad \text{velocità} \\
    D & \quad \text{diametro equivalente} \\
\end{align*}


\subsubsection{Numero di Nusselt}
\[ \Nus = \frac{h D}{k} \qty( = \frac{h \Delta T}{\frac{k}{D} \Delta T})\] 
Può essere interpretato come rapporto tra potenza termica scambiata per convezione e potenza termica scambiata per conduzione nello strato limite.

\subsubsection{Numero di Reynolds}
\[ \Rey = \frac{\rho w D}{\mu} \qty( = \frac{F_{inerzia}}{F_{viscose}})\] 
Può essere interpretato come rapporto tra la risultante delle forze di inerzia e la risultante delle forze viscose.

\textbf{Valori tipici di Reynolds}\\
\textit{Moto in un condotto:}\\
\phantom{→}$\put(3,2){\line(0,1){5}}\to$ laminare se $\Rey < 2000$, turbolento se $\Rey > 2500$\\
\textit{Moto lungo lastra piana:}\\
\phantom{→}$\put(3,2){\line(0,1){5}}\to$ laminare se $\Rey < 5 \cdot 10^5$, turbolento se $\Rey > 5 \cdot 10^5$\\
\textit{Moto attorno ad un cilindro:}\\
\phantom{→}$\put(3,2){\line(0,1){5}}\to$ laminare se $\Rey < 2 \cdot 10^5$, turbolento se $\Rey > 2 \cdot 10^5$\\

\subsubsection{Numero di Prandtl}
\[ \Pra = \frac{c_p \mu}{k} \]
Si ottiene dividendo $\nu = \frac{\mu}{\rho}$ (viscosità cinematica, $\nu$) per $a = \frac{k}{\rho c_p}$ (diffusività termica, $a$).

\textbf{Valori tipici di Prandtl}\\
\begin{tabular}{p{3cm}r}
    \toprule
    gas ideali & $0.7$\\
    acqua & \numrange{2}{10}\\
    metalli liquidi & \numrange{0.005}{0.03}\\
    \bottomrule
\end{tabular}\\

\subsection{Temperature di valutazione dei parametri}
\textbf{Corpo immerso}\\
\[ T_{film} = \frac{T_p + T_\infty}{2} \qq{oppure} T_{\text{asintotica}} = T_\infty \]
\textbf{Convenzione forzata in condotto}\\
\[ \qq*{Temp. di miscelamento adiabatico} T_f = \frac{\int_A \rho w c_p T \dd{A}}{\int_A \rho w c_p \dd{A}} \]

\subsection{Flusso all'interno di tubi}
In ogni sezione $w$ varia da un valore pari a zero sulla parete fino ad un valore massimo sull'asse del tubo.
In ogni sezione $T$ varia da un valore sulla parete fino ad un valore maggiore (o inferiore) sull'asse del tubo, a seconda se il processo sia di riscaldamento o raffreddamento.

\textbf{Velocità media}\\
La velocità media si ricava dal principio di conservazione della massa.
\[ w_m = \frac{\dot{m}}{\rho A_t} \qq{con} A_t = \pi \frac{D^2}{4} \]

\textbf{Temperatura media}\\
La temperatura media si ricava dal principio di conservazione dell'energia.
\[ \dot{E} = \dot{m} c_p T_m = \int_{\dot{m}} c_p T \partial \dot{m} = \int_{A_t} c_p T \qty(\rho w \dd{A_t}) \]

\textbf{Condizioni al contorno}\\
\[ T_s = \text{cost} \qq{oppure} \dot{q} = \text{cost} \]
Le due condizioni non possono essere contemporaneamente presenti.

\textbf{Flusso di calore costante}\\
\[ \dot{q}_s = \text{cost} \rightarrow \dot{Q}_s = \dot{q}_s \cdot A = \dot{m} c_p \qty(T_u - T_i) \]
Da cui deriva 
\[ T_u = T_i + \frac{\dot{q}_s A}{\dot{m} c_p} \]
Ma essendo $\dot{q}_s = h \qty(T_s - T_m)$
\[ T_s = T_m + \frac{\dot{q}_s}{h} \]
Nota come $T_m$ non è costante ma dipende dalla distanza dall'ingresso del tubo e si può calcolare usando l'equazione precedente che lega $T_u$ a $T_i$, considerando solo parte del tubo.

\textbf{Temperatura superficiale costante}\\
\[ \ln(\frac{T_s - T_u}{T_s - T_i}) = - \frac{h A}{\dot{m} c_p} \]
Ma anche
\[ \dot{Q} = h \cdot A \cdot \Delta T_{ml} \qq{con} \Delta T_{ml} = - \frac{T_u - T_i}{\ln(\frac{T_s - T_u}{T_s - T_i})} \]

\textbf{Caduta di pressione}\\
\[ \Delta P = f \cdot \frac{L}{D} \cdot \frac{\rho w_m^2}{2} \]
Dove $f$ è il \emph{fattore d'attrito}.
\[ \qq*{Potenza pompaggio} \dot{L}_p = \frac{\dot{m} \Delta P}{\rho} = w_m A_t \Delta P \]

\subsection{Convezione forzata}
\[ \Nus = A \cdot \Rey^\alpha \cdot \Pra^\beta \]
I parametri $\Rey$ e $\Pra$ vanno valutati alla giusta temperatura (che può essere di parete, asintotica, di film, di miscelamento adiabatico).

\subsubsection{Moto all'interno di un condotto circolare}
\begin{align*}
    f = \frac{64}{\Rey} & \quad \text{se laminare} \\
    f = 0.184 \Rey^{-0.2} & \quad \text{se turbolento}\\
\end{align*}
\begin{tabular}{ll}
    \toprule
    $\Nus_{D} = 3.66$ & laminare con $T_p = \text{cost}$ \\
    \midrule
    $\Nus_{D} = 4.36$ & laminare con $\dot{q} = \text{cost}$ \\
    \midrule
    \multirow{2}{*}{$\Nus_{D} = 0.023 \Rey^{0.8} \Pra^{0.3}$} & turbolento con $\Rey > 10^4$ \\
    & $0.7 < \Pra < 160$ (raffr.)\\
    \midrule
    \multirow{2}{*}{$\Nus_{D} = 0.023 \Rey^{0.8} \Pra^{0.4}$} & turbolento con $\Rey > 10^4$ \\
    & $0.7 < \Pra < 160$ (risc.)\\
    \midrule
    \multirow{2}{*}{$\Nus_{D} = 0.027 \Rey^{0.8} \Pra^{0.333} \qty(\frac{\mu}{\mu_p})^{0.14}$} & turbolento con $\Rey > 10^4$ \\
    & $0.7 < \Pra < 16700$\\
    \midrule
\end{tabular}\\
Le proprietà termofisiche sono valutate alla temperatura di miscelamento adiabatico.

\subsubsection{Moto attorno ad un cilindro}
Relazione di Hilpert
\[ \Nus_{D} = C \Rey^m \Pra^{\frac{1}{3}} \]
\begin{center}
\begin{tabular}{ccc}
    \toprule
    $\Rey$ & $C$ & $m$ \\ \midrule
    \numrange{0.4}{4} & 0.989 & 0.330 \\
    \numrange{4}{40} & 0.911 & 0.385 \\
    \numrange{40}{4000} & 0.683 & 0.466 \\
    \numrange{4000}{40000} & 0.193 & 0.618 \\
    \numrange{40000}{400000} & 0.027 & 0.805 \\
    \bottomrule
\end{tabular}
\end{center}
Le proprietà termofisiche sono valutate alla temperatura di film.

\subsubsection{Flusso su lastra piana interamente riscaldata}
Si suppone temperatura costante su tutta la piastra
\begin{tabular}{ll}
$\Nus_{L} = 0.664 \Rey^{0.5} \Pra^{\frac{1}{3}}$ & laminare $\Rey < 5\cdot10^5$\\
\multirow{2}{*}{$\Nus_{L} = 0.037 \Rey^{0.8} \Pra^{\frac{1}{3}}$} & turbolento $0.6 < \Pra < 60$\\
& $\quad 5 \cdot 10^5 < \Rey < 10^7$\\
\end{tabular}

\subsection{Convezione naturale}
\subsubsection{Numero di Grashof}
\[  \Gra = \frac{\rho^2 g \beta \Delta T D^3}{\mu^2} \]
Può essere interpretato come il rapporto fra le forze di galleggiamento ed il quadrato della risultante delle forze viscose.
\textbf{Moto lungo superficie verticale}\\
\begin{align*}
    \Gra_{L} < 10^9 & \quad \text{moto laminare} \\
    \Gra_{L} > 10^9 & \quad \text{moto turbolento} \\
\end{align*} 

\subsubsection{Numero di Peclet}
\[ \Pec = \Rey \cdot \Pra = \frac{w D}{a} \]

\subsubsection{Numero di Rayleigh}
\[ \Ray = \Gra \cdot \Pra = \frac{g \beta \Delta T D^3}{a \nu} \]

\subsubsection{Alcune correlazioni}

\begin{tabular}{|l|c|c|c|}
    \hline
    Geometria & $D$ & $\Ray$ & $\Nus$ \\
    \hline
    \multicolumn{4}{|l|}{Piastra} \\
    \hline
    Vert. altezza $L$ & $L$ & \numrange{e4}{e9} & $0.59 \Ray^{1/4}$ \\
     & & \numrange{e9}{e13} & $0.1 \Ray^{1/3}$ \\
    \hline
    Oriz. sopra calda & $\frac{A}{p}$ & \numrange{e4}{e7} & $0.54 \Ray^{1/4}$ \\
     & & \numrange{e7}{e11} & $0.15 \Ray^{1/3}$ \\
    \hline
    Oriz. sotto calda & $\frac{A}{p}$ & \numrange{e5}{e11} & $0.27 \Ray^{1/4}$ \\
    \hline
    \multicolumn{4}{|l|}{Cilindro} \\
    \hline
    Vert. alto L, diam D & $L$ & \multicolumn{2}{c|}{Come piast. vert. }\\
     & & \multicolumn{2}{c|}{quando $D \ge \frac{35L}{\Gra^{1/4}}$} \\
    \hline
    Oriz. diam D & $D$ & \numrange{e5}{e12} & $X$ \\
    \hline
\end{tabular}

Dove $p$ è il perimetro della piastra, \textbf{tutte le temperature} si riferiscono a $T_{\text{film}}$ e $X = \qty{0.60 + \frac{0.387 \Ray^{\frac{1}{6}}}{\qty[1 + \qty(\frac{0.559}{\Pra})^{\frac{9}{16}}]^{\frac{8}{27}}}}^2$.
