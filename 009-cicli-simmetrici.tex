\section{Cicli simmetrici}

Un ciclo formato da 4 politropiche, uguali a due a due.

\begin{minipage}{.6\linewidth}
    \begin{tikzpicture}[
        thick,
        >=stealth',
        dot/.style = {
        draw,
        fill = white,
        circle,
        inner sep = 0pt,
        minimum size = 4pt
        }
    ]
    \coordinate (O) at (0,0);
    \draw[->] (0,0) -- (4,0) coordinate[label = {below:$v$}] (xmax);
    \draw[->] (0,0) -- (0,4) coordinate[label = {left:$P$}] (ymax);

    \draw (3.5,0.5) node[below right] {4} parabola (1.5,1) node[below left] {1};
    \draw (1.5,1) parabola (0.5,3.5) node[above left] {2};
    \draw (2.5,3) parabola (0.5,3.5);
    \draw (3.5,0.5) parabola (2.5,3) node[above right] {3};

    \draw (1.5,3.2) node[above] {$m$};
    \draw (2,0.8) node[below] {$m$};
    \draw (0.8,2) node[left] {$n$};
    \draw (3.3,2) node[left] {$n$};
    \end{tikzpicture}
\end{minipage}%
\begin{minipage}{.35\linewidth}
    \begin{align*}
        v_1v_3 &= v_2v_4 \\
        P_1P_3 &= P_2P_4 \\
        T_1T_3 &= T_2T_4 \\
    \end{align*}
\end{minipage}

\subsection{Proprietà dei cicli simmetrici}

\begin{align*}
    P_1v_1^n &= P_2v_2^n \\
    P_2v_2^m &= P_3v_3^m \\
    P_3v_3^n &= P_4v_4^n \\
    P_4v_4^m &= P_1v_1^m \\
\end{align*}
