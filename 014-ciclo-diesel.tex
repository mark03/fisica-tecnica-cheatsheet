\section{Ciclo Diesel}

Ciclo \textbf{NON simmetrico} costituito da due \emph{isoentropiche}, una \emph{isobara} e una \emph{isocora}.
Applicazioni: motore endotermico.

\begin{minipage}{.5\linewidth}
\begin{tikzpicture}[thick,>=stealth']
    \coordinate (O) at (0,0);
    \draw[->] (0,0) -- (3.5,0) coordinate[label = {below:$s$}] (xmax);
    \draw[->] (0,0) -- (0,3.5) coordinate[label = {left:$T$}] (ymax);
    
    \draw (1,0.5) node[below left] {1} -- (1,1.25) node[above left] {2} parabola (3,3) node[above right] {3} -- (3,1.5) node[below] {4};
    \draw (1,0.5) parabola (3,1.5);
\end{tikzpicture}
\end{minipage}%
\begin{minipage}{.5\linewidth}
\begin{tikzpicture}[thick,>=stealth']
    \coordinate (O) at (0,0);
    \draw[->] (0,0) -- (3.5,0) coordinate[label = {below:$v$}] (xmax);
    \draw[->] (0,0) -- (0,3.5) coordinate[label = {left:$P$}] (ymax);

    \draw (2.75,0.75) node[below right] {1} [looseness=1.25,out=-180,in=-90] to (0.5,3) node[above left] {2} -- (1.25,3) node[above right] {3};
    \draw (2.75,1.5) node[right] {4} [looseness=1.25,out=-180,in=-90] to (1.25,3);
    \draw (2.75,0.75) -- (2.75,1.5);
\end{tikzpicture}
\end{minipage}

Nell'ipotesi di \emph{gas perfetto} e \emph{ciclo ideale}:
\[ \eta_{diesel} = \frac{L}{Q_C} = 1 - \frac{c_v (T_4-T_1)}{c_p (T_3-T_2)} = 1 - \frac{c_v T_1 (\frac{T_4}{T_1}-1)}{c_p T_2 (\frac{T_3}{T_2}-1)} \]

\begin{align*}
\text{Rapporto compressione volumetrico:} &\qquad r = \frac{V_1}{V_2} \\
\text{Rapporto di combustione:} &\qquad z = \frac{V_3}{V_2} \\
\end{align*}

\[ \eta_{diesel} = 1 - \frac{1}{r^{k-1}} \frac{1}{k} \frac{z^k-1}{z-1} \]

\subsection{Confronto ciclo Otto - ciclo Diesel}
\begin{itemize}
    \item Motore Otto è più leggero a parità di potenza
    \item Motore Otto ha frequenza di rotazione maggiore
    \item Motore Otto ha minore rumorosità
    \item Motore Diesel ha miglior rendimento a causa del maggior rapporto di compressione (circa il doppio rispetto a motore Otto)
    \item Motore Diesel ha miglior rendimento al diminuire del carico (più facilmente regolabile in potenza)
    \item Motore Diesel utilizza un combustibile meno pregiato del motore Otto
\end{itemize}
