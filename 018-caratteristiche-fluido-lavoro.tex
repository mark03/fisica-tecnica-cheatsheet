\section{Fluidi di lavoro in cicli a vapore}

\begin{itemize}
    \item Elevata massa volumica e entalpia di transizione di fase $\Rightarrow$ ridurre la portata di fluido;
    \item Elevata temperatura critica $\Rightarrow$ al punto critico $h_{lvt} = 0$;
    \item Temperatura del punto triplo inferiore a quella minima del ciclo $\Rightarrow$ evitare fase solida;
    \item Fluido non corrosivo $\Rightarrow$ ridurre i costi e manutenzione;
    \item Fluido non tossico $\Rightarrow$ ridurre i rischi ambientali;
    \item Chimicamente stabile $\Rightarrow$ aumentare sicurezza impianto;
    \item Facilmente reperibile e di basso costo;
    \item Elevata pendenza della curva limite superiore in $T-s$ $\Rightarrow$ vapore in uscita con elevato titolo;
    \item Pressione di condensazione $> \SI{1}{atm}$ $\Rightarrow$ evitare infiltrazioni di gas incondensabili;
\end{itemize}

Esempi di fluidi adatti:

\begin{itemize}
    \item Ciclo motore: acqua
    \item Ciclo frigorifero:
    \begin{itemize}
        \item Ammoniaca $\ch{NH3}$ $\Rightarrow$ tossico;
        \item Clorofluorocarburi ($\ch{CFC}$ o freon) $\Rightarrow$ dannosi per l'ozono;
        \item Clorofluoroidrocarburi ($\ch{HCFC}$) o fluoroidrocarburi ($\ch{HFC}$) $\Rightarrow$ meno dannosi per l'ozono ($R134a$);
    \end{itemize}
\end{itemize}
